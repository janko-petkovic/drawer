\documentclass[a4paper, 10pt]{article}

% Encoding
\usepackage[T1]{fontenc}
\usepackage[utf8]{inputenc}
\usepackage[english]{babel}

% Margins
\usepackage[left=2.5cm, right=2.5cm]{geometry}

% Mathematics
\usepackage{amsmath, amssymb, mathtools, amsthm}

% Bibliography (uncomment if needed)
\usepackage{csquotes}
\usepackage[backend=biber, sorting=none, style=authoryear]{biblatex}
\addbibresource{biblio.bib}

% Personal taste
\usepackage[font=footnotesize]{caption}
\usepackage[parfill]{parskip}


\title{Inferring cerebellar computations with probabilistic machine learning
approaches}
\author{}

\begin{document}

\maketitle
% Locomotion is one of the strategies that biological systems use to interact with
% the surrounding environment. In some organisms a specific unit, the cerebellum,
% has evolved to achieve a high degree of motor coordination during the execution
% of motor programs. How this function is implemented in the cerebellar circuitry
% and what are the theoretical principles driving its emergence are still active
% fields of research.

Growing evidence is suggesting that vertebrate adaptability relies on the
cerebellum's capacity to channel information well beyond its traditionally known
motor pathways. In particular, researchers in the field have discovered novel
projections to sub-cortical and reward related regions (\cite{Washburn2024TheCD,
Ohmae2015ClimbingFE, Washburn2024TheCD, Wagner2017CerebellarGC}), observing that
the cerebellum is involved in a plethora of tasks not directly related to
movement tuning (\cite{Strick2009CerebellumAN, Overwalle2014SocialCA}). Despite
this improvement in phenomenological understanding, the grounding theoretical
principles driving cerebellar computations are still strongly debated.

To address this question, several computational models have been proposed,
dissecting specific functional aspects and leveraging different mathematical
frameworks (\cite{DeSchepper2021ModelSU, DAngelo2016ModelingTC,
Diedrichsen2019UniversalTO}). We plan to build on top of this work and propose a
unified theoretical framework of cerebellar computation by integrating model
selection and inference theory with experimental validation and neuromorphic
applications.

We will start by studying the equivalence of the proposed models with respect to
their emerging behaviour, framing the question as a model degeneracy (or
robustness) problem (\cite{Calaim2022TheGO, Gonalves2019TrainingDN}). To tackle
this problem, we will start by characterizing the models' parameter space using
simulation-based inference (\cite{Lueckmann2017FlexibleSI}), leveraging the
latest toolkit developments achieved in our group (\cite{TejeroCantero2020sbiAT,
Deistler2024DifferentiableSE}). Coupling this approach with summary statistic
design and dimensionality reduction techniques
(\cite{Pellegrino2024DimensionalityRB, Cenedese2022DatadrivenMA}) we will
characterize these parameter spaces in terms of their computational features,
low-dimensional dynamics, and other optimality metrics such as energy efficiency
(\cite{Jedlicka2022ParetoOE}). With the aid of variational inference techniques
(\cite{Bishop2012PatternRA, Luo2022UnderstandingDM}), we will search for common
latent factors that underpin these emerging features across different models,
ultimately separating a healthy neuronal circuit from a pathological one.

Building on these insights, we will develop a new computational model, and test
the predictions of our theoretical framework on existing human and animal
cerebellar datasets (\cite{Kumar2022PhysiologicalRO, 10.1242/dmm.049514}).
Moreover, in strict collaboration with experts in the field
(\cite{Ribar2020NeuromorphicCD}), we will investigate how the newly discovered
factors can be applied to neuromorphic networks to achieve a superior degree of
robustness in motor control tasks.

In summary, by bridging advanced parameter inference, model selection and
experimental data analysis, this project will unveil the foundational principles
underlying cerebellar computations. This will ultimately pave the way to novel
diagnostic approaches and resilient, brain-inspired computational motor
controllers.




\printbibliography

\end{document}
