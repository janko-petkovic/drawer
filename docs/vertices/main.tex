\documentclass[a4paper, 10pt]{article}

% Encoding
\usepackage[T1]{fontenc}
\usepackage[utf8]{inputenc}
\usepackage[english]{babel}

% Margins
\usepackage[left=3cm, right=3cm]{geometry}

% Mathematics
\usepackage{amsmath, amssymb, mathtools, amsthm}
\usepackage{cancel}

% Bibliography (uncomment if needed)
%\usepackage{csquotes}
%\usepackage[backend=biber, sorting=none, style=nature]{biblatex}
%\addbibresource{biblio.bib}

% Personal taste
\usepackage[font=footnotesize]{caption}
\usepackage[parfill]{parskip}


\title{vertices}
\author{}

\begin{document}

\maketitle

\subsection{Merger}

Consider a weighted complete graph $G(V, E)$ with a set of vertices $V$ and a
set of edges $E$ formed by pairs of elements in $V$. The set $V$ has cardinality
$|V| = N$ and the set $E$ has cardinality $|E| = N(N-1)/2$. The latter number
will appear often, so we will denote it compactly as $T_{N-1}$, the $N-1$th
triangular number. We will denote by $w_{ij} = w(v_i, v_j)$ the weight of the
edge connecting the vertices $i$ and $j$.

Choosing a subset $A \subset V$ of cardinality $k$ induces a partition in the edges:

\begin{equation}
    E = E_A \cup E_{AA^c} \cup E_{A^c}
\end{equation}

Where $E_A$ are the edges formed by the complete subgraph $G(A, E_A)$. The 
elements of this set are called the intra-edges of $A$. The set $E_{AA^c}$
represents the edges with one vertex in $A$ and another in its complement $A^c$,
and its elements are called the inter-edges of $A$:

\begin{align}
    E_A &:= \{\{v_i, v_j\} \in E | v_i \in A \land v_j \in A\} \\
    E_{AA^c} &:= \{\{v_i, v_j\} \in E | (v_i \in A \land v_j \in A^c) \lor (a \in A^c \land b \in A)\}
    \label{eq:intra-inter-edge-def} 
\end{align}
.

Since $G(A, E_A)$ is complete, the cardinality of $E_A$ is $T_{k-1}$ whereas the cardinality of $E_{AA^c}$ is $(N-k)k$, since

\begin{align}
    |E| &= |E_A \cup E_{AA^c} \cup E_{A^c}| = |E_A| + |E_{AA^c}| + |E_{A^c}|\\
    T_{N-1} &= T_{k-1} + |E_{AA^c}| + T_{N-k-1} \\
    |E_{AA^c}| &= (N-k)k
\end{align}
.

With this construction, our goal is to compute the expected value of the average of weights of $E_A$ provided we choose the vertices of $A$ with uniform probability, that is

\begin{equation}
    \mathbb{E}\left[\frac{1}{|E_A|}\sum_{e \in E_A} w(e) \Big\vert |A| = k\right]
    \label{eq:expected-average-def}
\end{equation}
.

From this point, it will be convenient to enumerate the set of vertices with a natural ordering $V = \left\{v\right\}_{i=1}^N$, so we can rewrite equation \ref{eq:expected-average-def} as

\begin{equation}
    \frac{1}{T_{k-1}}\mathbb{E}\left[\sum_{\substack{i>j \\ v_i,v_j \in A}}w((v_i,v_j)) \Big\vert |A| = k\right]
    \label{eq:expected-average-vertex}
\end{equation}
.

Since the choice of vertices completely define the outcome, we define the set of events $\mathcal{U}$ to be a sigma algebra over the vertices as the power set of $V$:

\begin{equation}
    \mathcal{U} = 2^{V}
    \label{eq:power-set-is-sigma-algebra}   
\end{equation}
.

We are particularly interested in the subset $\mathcal{U}_k \coloneq \{A \in \mathcal{U} | |A|=k\}$ with probability

\begin{equation}
    p_k \coloneq \mathbb{P}(A|\mathcal{U}_k) = \frac{1}{\binom{N}{k}}
    \label{eq:conditional-probability-fixed-k}
\end{equation}
.

Now, we can expand equation \ref{eq:expected-average-vertex} as

\begin{equation}
    \frac{1}{T_{k-1}}\mathbb{E}\left[\sum_{\substack{i>j \\ v_i,v_j \in A}}w((v_i,v_j)) \Big\vert |A| = k\right] = \frac{1}{T_{k-1}}\sum_{A \in \mathcal{U}_k} p_k \sum_{\substack{i>j \\ v_i,v_j \in A}}w((v_i,v_j))
    \label{eq:expectation-as-sum}
\end{equation}
.

The rightmost sum can be simplified with the use of indicator functions:

\begin{equation}
    \chi_{A}(v) = \begin{cases}
        1, & v \in A \\
        0
    \end{cases}
    \label{eq:indicator-function}
\end{equation}

Which allows to rewrite the sum as

\begin{equation}
    \sum_{\substack{i>j \\ v_i,v_j \in A}}w((v_i,v_j)) = \sum_{\substack{i>j \\ v_i,v_j \in V}}\chi_A(v_i)\chi_A(v_j)w((v_i,v_j))
\end{equation}

Resulting in equation \ref{eq:expectation-as-sum} to become

\begin{equation}
    \frac{1}{T_{k-1}}\sum_{A \in \mathcal{U}_k} p_k \sum_{\substack{i>j \\ v_i,v_j \in A}}w((v_i,v_j)) = \frac{p_k}{T_{k-1}}\sum_{\substack{i>j \\ v_i,v_j \in V}}w((v_i,v_j))\sum_{A \in \mathcal{U}_k}\chi_A(v_i)\chi_A(v_j)
    \label{eq:expectation-smaller}
\end{equation}
.

The rightmost sum is a count over how many times the edge $\{v_i,v_j\}$ appears in different sets of $\mathcal{U}_k$, more specifically, we look for the cardinality of the set $\mathcal{U}_k(v_i,v_j)$ defined as

\begin{equation}
    \mathcal{U}_k(v_i, v_j) \coloneq \{A \in \mathcal{U}_k | A \cap \{v_i, v_j\} \neq \emptyset\}
\end{equation}

Which is isomorphic to

\begin{equation}
    \mathcal{U}'_k(v_i, v_j) \coloneq \{A - \{v_i, v_j\} | A \in \mathcal{U}_k\} - \{\emptyset\}
\end{equation}

With cardinality $\binom{N-2}{k-2}$. Replacing this value with the sum in equation \ref{eq:expectation-smaller} yields

\begin{align}
    \mathbb{E}\left[\frac{1}{T_{k-1}}\sum_{e \in E_A} w(e) \Big\vert |A| = k\right] = \dots &= \frac{p_k}{T_{k-1}}\sum_{\substack{i>j \\ v_i,v_j \in V}}w((v_i,v_j))\binom{N-2}{k-2} \\
    &= \frac{\cancel{p_k}}{T_{k-1}}\sum_{\substack{i>j \\ v_i,v_j \in V}}w((v_i,v_j))\cancel{\binom{N}{k}}\frac{T_{k-1}}{T_{N-1}} \\
    &= \frac{1}{T_{N-1}}\sum_{\substack{i>j \\ v_i,v_j \in V}}w((v_i,v_j))
\end{align}

.
  % \printbibliography

\end{document}
